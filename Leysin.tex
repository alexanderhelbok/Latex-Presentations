\documentclass[10pt]{beamer}
\usepackage[utf8x]{inputenc}
\usepackage{hyperref}
\usepackage{fontawesome5}
\usepackage{graphicx}
\usepackage[english,ngerman]{babel}
\usepackage{setspace}
\usepackage{multicol, enumitem}

\graphicspath{ {./Graphics/} }
\setlist[enumerate]{label=\arabic*.}
% ------------------------------------------------------------------------------
% Use the beautiful metropolis beamer template
% ------------------------------------------------------------------------------
\usepackage[T1]{fontenc}
\usepackage{fontawesome5}
\usepackage{FiraSans} 
\mode<presentation>
{
	\usetheme[progressbar=foot,numbering=fraction,background=light]{metropolis} 
	\usecolortheme{default} % or try albatross, beaver, crane, ...
	\usefonttheme{default}  % or try serif, structurebold, ...
	\setbeamertemplate{navigation symbols}{}
	\setbeamertemplate{caption}[numbered]
	%\setbeamertemplate{frame footer}{My custom footer}
} 
\setitemize{label=\usebeamerfont*{itemize item}%
	\usebeamercolor[fg]{itemize item}
	\usebeamertemplate{itemize item}}

% ------------------------------------------------------------------------------
% tcolorbox / tcblisting
% ------------------------------------------------------------------------------
\usepackage{xcolor}
\definecolor{codecolor}{HTML}{FFC300}

\usepackage{tcolorbox}
\tcbuselibrary{most}

\tcbset{tcbox width=auto,left=0mm,top=1mm,bottom=1mm,
	right=0mm,boxsep=1mm,middle=1pt}

\newtcolorbox{myr}{colback=codecolor!5,colframe=codecolor!80!black,
	left=3mm, right=3mm, enhanced, fonttitle=\bfseries}


\title{Winkeltreue im
	Diskreten und ihre Anwendung in der Physik}
\author{Alexander Helbok, Helen, Patryk, Matthias}
\date{\today}

\begin{document}
	\setbeamertemplate{caption}{\raggedright\insertcaption\par}
	
	\maketitle
	\begin{frame}[fragile]{Überblick}
		\setstretch{1.5}
		\begin{enumerate}
			\item Recap Holomorphe Abbildungen
			\item Konforme/Winkeltreue Abbildunen 
			\item Symmetrie in der Physik 
			\item Exkurs: Lagrangian 
			\item Nöthers Theorem
		\end{enumerate}
	\end{frame}

	\begin{frame}{Holomorphe Abbildungen}
		\vspace{1cm}\begin{itemize}[label={--}]
			\item komplexe Funktion
			\item komplex diffbar \\
			\( \Rightarrow \) unendlich diffbar + analytisch \\
			\( \Rightarrow \) sehr schöne Funktion
			\item Komposition bleibt Holomorph 
		\end{itemize}
		\hspace{6.25cm}\begin{minipage}{0.4\textwidth}
			\vspace{-3.5cm}\begin{figure}[h]
				\only<1>{\includegraphics[scale=0.35]{Ref}}
				\only<1>{\caption{\( \operatorname{Re}(\Gamma) \)}}
				\only<2>{\includegraphics[scale=0.35]{Imf}}
				\only<2>{\caption{\( \operatorname{Im}(\Gamma) \)}}
			\end{figure}
		\end{minipage}
	\end{frame}

	\begin{frame}{Konforme/Winkeltreue Abbildungen}
		\begin{itemize}[label={--}]
			\item bijektive Funktionen, die Winkel im infinitesimalen erhalten 
			\item oder: Transformation, deren Jacobi-Matrix ein Skalar * Rotationsmatrix (mit Det = 1) ist
			\item Konforme Abbildungen sind lokal invertierbare komplex analytische Funktionen (Holomorphe Funktionen)5
		\end{itemize}
		\begin{minipage}{\textwidth}
			\hspace{0.5cm}
			\includegraphics[scale=1.5]{ang1}
			\hspace{1cm}
			\begin{minipage}{0.1\textwidth}
				\vspace{-3cm}\Huge\( \Rightarrow \)
			\end{minipage}
			\hspace{1cm}
			\includegraphics[scale=2]{ang3}
		\end{minipage}
	\end{frame}
	\begin{frame}{Exkurs: Lagrangian}
		"To make clear why this is important for physics, we remind that the laws of nature do not depend on the coordinates we use
		to describe them. On the other hand, physicists cannot formulate laws without coordinates, and physical equations usually look different in different coordinates. This is why the relationship between physical theories and coordinates should be as well-defined and restricted
		as possible. The Lagrangian formalism fulfills this demand through the Euler-Lagrange equations being independent of coordinate transformations. Lagrangian functionals do depend on coordinates, but in the simplest way physicists can think of: they transform like scalars."
	\end{frame}

	\begin{frame}{Nöthers Theorem}
		\centering
		Gibt es einen Zusammenhang zwischen Symmetrie und Erhaltungsgrößen?\\[1cm]
		\uncover<2-3>{\Large Ja!\\[1cm]}
		\uncover<3-3>{\normalsize
		\begin{myr}
			„If a system has a continuous symmetry property, then there are corresponding quantities whose values are conserved in time.”
		\end{myr}}
	\end{frame}
\end{document} 
