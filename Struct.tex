\documentclass[10pt]{beamer}
\usepackage[utf8x]{inputenc}
\usepackage{hyperref}
\usepackage{fontawesome5}
\usepackage{graphicx}
\usepackage[english,ngerman]{babel}
\usepackage{setspace}

% ------------------------------------------------------------------------------
% Use the beautiful metropolis beamer template
% ------------------------------------------------------------------------------
\usepackage[T1]{fontenc}
\usepackage{fontawesome5}
\usepackage{FiraSans} 
\mode<presentation>
{
	\usetheme[progressbar=foot,numbering=fraction,background=light]{metropolis} 
	\usecolortheme{default} % or try albatross, beaver, crane, ...
	\usefonttheme{default}  % or try serif, structurebold, ...
	\setbeamertemplate{navigation symbols}{}
	\setbeamertemplate{caption}[numbered]
	%\setbeamertemplate{frame footer}{My custom footer}
} 

% ------------------------------------------------------------------------------
% minted
% ------------------------------------------------------------------------------
\usepackage[outputdir=build]{minted}


% ------------------------------------------------------------------------------
% tcolorbox / tcblisting
% ------------------------------------------------------------------------------
\usepackage{xcolor}
\definecolor{codecolor}{HTML}{FFC300}

\usepackage{tcolorbox}
\tcbuselibrary{most,listingsutf8,minted}

\tcbset{tcbox width=auto,left=0mm,top=1mm,bottom=1mm,
	right=0mm,boxsep=1mm,middle=1pt}

\newtcblisting{myr}{colback=codecolor!5,colframe=codecolor!80!black,listing only, 
	minted options={numbers=left,fontsize=\small,breaklines,autogobble,linenos,numbersep=3mm, tabsize=3},
	left=5mm,enhanced, fonttitle=\bfseries, 
	listing engine=minted,minted language=C}

\newtcblisting{tmyr}{colback=codecolor!5,colframe=codecolor!80!black,listing only, 
	minted options={numbers=left,fontsize=\scriptsize ,breaklines,autogobble,linenos,numbersep=3mm, tabsize=3},
	left=5mm,enhanced, fonttitle=\bfseries, 
	listing engine=minted,minted language=C}


\title{Structs}
\author{Alexander Helbok, Raphael Riener, Anna Bozukov, Lorenz Ritsch}
\date{\today}

\begin{document}
	
	\maketitle
	\begin{frame}[fragile]{Überblick}
		\setstretch{1.5}
		\begin{enumerate}
			\item Definition
			\item Initialisierung
			\item Nested Structs
			\item Structs und Funktionen
			\item Structs und Pointer
			\item Motivation
			\item Bsp Rakete
		\end{enumerate}
	\end{frame}
	
	\begin{frame}[fragile]{Einleitung}
		\Large Was sind Structs und warum brauchen wir sie?
	\end{frame}

	\begin{frame}[fragile]{Definition}
		\begin{myr}
			struct person{
				char name[50];
				int age;
				float height;
			}; 
		
			int main(){
				struct person Alex, Anna;
			}\end{myr}\\
		\begin{myr}
			struct person{
				char name[50];
				int age;
				float height;
			} Alex, Anna; \end{myr}
	\end{frame}

	\begin{frame}[fragile]{Initialisierung}
		\begin{myr}
			// Struct declaration
			
			int main(){
				Alex.age = 19;
				Anna.height = 1.85;
				// Ausgabe
				printf("Alter = %d\n", Alex.age);
			} \end{myr}
		\begin{myr}
			// Struct declaration
			
			int main(){
				struct person Alex = {.height = 1.7, .age = 19};
				struct person Anna = {"Anna", 20, 1.6};
				// Ausgabe
				printf("Alter = %d\n", Alex.age);
			} \end{myr}
	\end{frame}

	\begin{frame}[fragile]{Nested Structs}
		\begin{myr}
			struct person{
				char name[50];
				int age;
				float height;
			}; 
			
			struct Family{
				struct person;
				int number_Sisters;
				int number_Brothers;
			} myFamily;
			
			int main(){
				struct person Alex = {.age = 19, .height = 1.7};
				struct Family myFamily = {Alex, .number_Sister = 2};
				printf("Height = %f\n", myFamily.Alex.height);
			}\end{myr}
	\end{frame}
	
	\begin{frame}[fragile]{Structs und Funktionen}
		\begin{myr}
		// Struct declaration
		
		struct person avgPerson(struct person p1, struct person p2);
		
		int main(){
			struct person Alex = {.age = 19, .height = 1.7};
			struct person Anna = {.age = 20, .height = 1.6};
			struct person Avg = avgPerson(Alex, Anna);
		}
	
		struct person avgPerson(struct person p1, struct person p2){
			struct person temp;
			temp.age = (p1.age + p2.age)/2;
			temp.height = (p1.height + p2.height)/2;
			return temp;
		}\end{myr}
	\end{frame}

	\begin{frame}[fragile]{Structs und Pointer}
		\begin{myr}
		// Struct declaration
		
		void aging(struct person *p1);
		
		int main(){
			struct person Alex = {.age = 19, .height = 1.7};
			aging(&Alex);
		}
		
		void aging(struct person *p1){
			p1->age += 50;
			(*p1).height -= 0.03;
		}\end{myr}
	\end{frame}

	\begin{frame}{Motivation}
		\setstretch{1.5}
		\begin{enumerate}
			\item Datentypen mischen
			\item Strukturierter als Arrays
			\item Vereinfachte nachträgliche Änderung
			\item Benennung der Felder macht Abfrage einfacher
			\item Verschachteln 
		\end{enumerate}
	\end{frame}
	
	\begin{frame}{Bsp Rakete}
		\begin{center}
			\Large Reichweite und Flugdauer einer Rakete \\[1ex]
			mit benutzerdefinierten Werten für \\[1ex]
			Treibstoffmasse, Masseverlustrate usw.
		\end{center}
	\end{frame}

	\begin{frame}[fragile]{Bsp Rakete}
		\begin{tmyr}
		struct rakete {
			double m;		// Rakete Leermasse
			double v;		// Rakete Geschwindigkeit
			
			double mTank;	// Treibstoff Masse
			double vTank;	// Treibstoff Geschwindigkeit
			double dmTank;	// Treibstoff Masseverlustrate
			
			double s;		// Zurückgelegte Strecke
			double t;		// Vergangene Zeit
		};\end{tmyr}\\
		\begin{tmyr}
		void masseschritt(struct rakete * rocket, double delta_m) {
			rocket->v += rocket->vTank * delta_m/(rocket->m + rocket->mTank);
			rocket->mTank -= delta_m;    
		}\end{tmyr}\\
		\begin{tmyr}
		void zeitschritt(struct rakete * rocket, double delta_t) {
			rocket->s += rocket->v * delta_t;
			rocket->t += delta_t;
		}\end{tmyr}
	\end{frame}

	\begin{frame}[fragile]{Bsp Rakete}
		\begin{tmyr}
			struct rakete prototyp = {.v = 0, .s = 0, .t = 0,
				// restliche variablen via argv[] zuweisen
			};\end{tmyr}\\
		\begin{tmyr}
			struct rakete rocket = prototyp;
			double delta = 1e-5;\end{tmyr}\\
		\begin{tmyr}
			while (rocket.mTank > 0) { 
				masseschritt(&rocket, delta);
				zeitschritt(&rocket, delta/rocket.dmTank);
			}\end{tmyr}
		\begin{tmyr}
			printf("Endgeschwindigkeit %f erreicht nach: %f\n", rocket.v, rocket.t);\end{tmyr}
	\end{frame}

	\begin{frame}{}
		\begin{center}
			\Large Danke für Ihre Aufmerksamkeit!
		\end{center}
	\end{frame}
	
\end{document}